% A2AX-Core Protocol — Informational Overview (Companion Document)
% Release v0.1.3 | Protocol Specification v1.0 | MIT License | Williams Creative | February 2026
%
% Companion to: 001 A2AX-Core Open Trust Standard (Main Whitepaper), 003 A2AX-Core Protocol Specification (Normative)

\documentclass[11pt,a4paper]{article}
\usepackage[utf8]{inputenc}
\usepackage[T1]{fontenc}
\usepackage{hyperref}
\usepackage{enumitem}
\usepackage[margin=1in]{geometry}
\usepackage[parfill]{parskip}

\hypersetup{colorlinks=true, linkcolor=blue, urlcolor=blue, citecolor=blue}

\title{A2AX-Core Protocol\\Informational Overview}
\author{Williams Creative}
\date{Companion Document \\ Version 0.1.3}

\begin{document}

\maketitle

\begin{abstract}
This document provides an informational overview of the A2AX-Core Protocol. It is not normative. The normative specification is defined in the A2AX-Core Protocol Specification (003).
\end{abstract}

\tableofcontents
\newpage

\section{Purpose}

A2AX-Core defines a neutral trust substrate for autonomous agents. It enables portable identity, verifier-controlled trust decisions, and secure agent-to-agent coordination without centralized registries or embedded authority.

This document is informational. The normative specification is defined in the A2AX-Core Protocol Specification.

\section{Why A2AX-Core Exists}

Autonomous agents increasingly coordinate across systems, organizations, and jurisdictions. Intelligence alone does not ensure safe interaction. A portable and cryptographically verifiable trust layer is required.

A2AX-Core addresses this gap by defining minimal, implementation-neutral trust primitives.

\section{Core Properties}

\begin{itemize}[leftmargin=*]
    \item Portable identity certificates
    \item Verifier-controlled trust model
    \item No mandatory trust anchors
    \item No centralized registry requirement
    \item Capability-scoped permissions
    \item Secure agent-to-agent handshake
\end{itemize}

The protocol is infrastructure---not marketplace, governance system, or economic layer.

\section{Layered Architecture}

\begin{quote}
Application / Agent Layer \\
$\downarrow$ \\
A2AX-Core Protocol Layer (Identity + Verification + Handshake) \\
$\downarrow$ \\
Verifier Trust Policy Layer \\
$\downarrow$ \\
Optional Extensions (Economic, Compliance, Analytics)
\end{quote}

The core remains minimal and neutral.

\section{What A2AX-Core Is Not}

A2AX-Core does not define:

\begin{itemize}[leftmargin=*]
    \item Token systems
    \item Market mechanisms
    \item Economic valuation
    \item Governance frameworks
    \item Centralized infrastructure
\end{itemize}

Higher-order systems may build on A2AX-Core but remain external to it.

\section{Neutrality Principle}

Trust is determined exclusively by the verifier.

No global root of trust exists.

No organization controls protocol validity.

Neutrality is structural, not rhetorical.

\section{Intended Audience}

\begin{itemize}[leftmargin=*]
    \item Protocol implementers
    \item Multi-agent system architects
    \item Infrastructure engineers
    \item Standards and governance bodies
\end{itemize}

\section{Relationship to the Specification}

The A2AX-Core Protocol Specification defines normative requirements using formal conformance language (MUST, SHOULD, etc.).

This overview provides architectural context and positioning clarity.

\section{Long-Term Vision}

A2AX-Core aims to serve as a portable, minimal trust standard for autonomous agent coordination across ecosystems.

Its durability depends on neutrality, verifier sovereignty, and strict scope boundaries.

\section{Threat Model (Informational Summary)}

A2AX-Core assumes the presence of capable adversaries operating at the network, agent, and ecosystem levels.

The protocol is designed to mitigate:

\begin{itemize}[leftmargin=*]
    \item Certificate forgery attempts
    \item Message replay attacks
    \item Capability escalation attempts
    \item Malicious but cryptographically valid agents
    \item Centralization pressure through mandatory trust anchors
\end{itemize}

A2AX-Core does \textbf{not} attempt to mitigate:

\begin{itemize}[leftmargin=*]
    \item Economic fraud
    \item Marketplace manipulation
    \item Behavioral dishonesty
    \item Global revocation enforcement
\end{itemize}

The protocol guarantees cryptographic authenticity, not behavioral integrity.

\section{Adversary Classes}

\textbf{Class A --- Network Adversary}

May intercept, replay, or inject traffic.

\textbf{Class B --- Malicious Certified Agent}

Possesses valid credentials but behaves dishonestly.

\textbf{Class C --- Compromised Key Holder}

Private key material has been exposed.

\textbf{Class D --- Centralization Actor}

Attempts to impose mandatory trust anchors or ecosystem control.

Mitigation across classes is achieved through signature validation, nonce enforcement, expiration limits, and strict verifier sovereignty.

\section{Separation of Concerns}

A2AX-Core enforces structural separation between four distinct concepts.

\subsection{Identity}

Cryptographic binding between an agent identifier and a public key.

Answers: \textit{Who signed this message?}

\subsection{Capability}

Declared functional scope cryptographically bound to identity.

Answers: \textit{What does this agent claim it can do?}

Capabilities are claims, not guarantees.

\subsection{Trust}

A verifier-local decision derived from policy.

Answers: \textit{Do I accept interaction under my rules?}

Trust is never embedded in the certificate.

\subsection{Reputation (Out of Scope)}

Accumulated behavioral or economic history over time.

Answers: \textit{How has this agent behaved historically?}

Reputation is explicitly external to A2AX-Core and MUST NOT be embedded within the core protocol layer.

\end{document}
