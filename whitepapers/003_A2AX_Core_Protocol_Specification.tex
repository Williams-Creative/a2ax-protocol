% A2AX-Core Protocol — Normative Specification
% Release v0.1.3 | Protocol Specification v1.0 | MIT License | Williams Creative | February 2026
%
% Companion to: 001 A2AX-Core Open Trust Standard (Main Whitepaper), 002 A2AX-Core Overview (Informational)

\documentclass[11pt]{article}
\usepackage[a4paper, margin=1in]{geometry}
\usepackage[utf8]{inputenc}
\usepackage[T1]{fontenc}
\usepackage{hyperref}
\usepackage{enumitem}
\usepackage{longtable}
\usepackage{booktabs}
\usepackage{courier}

\hypersetup{colorlinks=true, linkcolor=blue, urlcolor=blue, citecolor=blue}

\title{A2AX-Core Protocol Specification}
\author{Williams Creative}
\date{Intended Status: Standards Track \\ Version 0.1.3}

\begin{document}

\maketitle

\begin{abstract}
This document defines the normative specification of the A2AX-Core Protocol. It standardizes portable identity certificates, cryptographic verification requirements, capability declaration semantics, trust store behavior, and an agent-to-agent handshake mechanism. The protocol is neutral, verifier-controlled, and registry-independent.
\end{abstract}

\tableofcontents
\newpage

\section{Status of This Document}

This document specifies version 0.1.3 of the A2AX-Core Protocol.

The key words ``MUST'', ``MUST NOT'', ``REQUIRED'', ``SHALL'', ``SHOULD'', ``SHOULD NOT'', ``RECOMMENDED'', ``MAY'', and ``OPTIONAL'' are to be interpreted as described in RFC 2119.

\section{Terminology}

\textbf{Agent} --- Autonomous computational system capable of initiating interactions.

\textbf{Identity Certificate} --- Self-contained cryptographically signed artifact binding an Agent identifier to capabilities.

\textbf{Issuer} --- Entity that signs Identity Certificates.

\textbf{Verifier} --- Entity validating certificates and applying local trust policy.

\textbf{Trust Store} --- Local collection of trusted issuer public keys.

\section{Separation of Concerns}

A2AX-Core enforces strict structural separation between identity, capability, trust, and reputation.

\subsection{Identity}

Identity is the cryptographic binding between an agent identifier and a public key.

Identity answers: ``Who signed this message?''

Identity MUST NOT imply competence, reliability, honesty, or economic value.

\subsection{Capability}

Capabilities are declared functional scopes cryptographically bound to identity.

Capability answers: ``What does this agent claim it can do?''

Capabilities are claims and MUST NOT be interpreted as guarantees.

\subsection{Trust}

Trust is a verifier-local policy decision.

Trust answers: ``Do I accept interaction under my rules?''

Trust MUST remain exclusively under verifier control and MUST NOT be embedded within certificates.

\subsection{Reputation (Out of Scope)}

Reputation represents accumulated behavioral or economic history over time.

Reputation systems MAY exist externally but MUST NOT be embedded into the A2AX-Core protocol layer.

The core protocol SHALL NOT define scoring, ranking, weighting, or economic valuation mechanisms.

\section{Protocol Overview}

The protocol defines:

\begin{itemize}[leftmargin=*]
    \item Identity Certificate structure
    \item Verification algorithm
    \item Trust store requirements
    \item Capability declaration semantics
    \item Agent-to-Agent (A2A) handshake procedure
\end{itemize}

\section{Identity Certificate Structure}

An Identity Certificate MUST include:

\begin{itemize}[leftmargin=*]
    \item Agent identifier
    \item Issuer identifier
    \item Agent public key (Ed25519)
    \item Capability declarations
    \item Issuance timestamp
    \item Expiration timestamp
    \item Digital signature (Ed25519)
\end{itemize}

Certificates MUST be self-contained and MUST NOT require registry lookup for validation.

\section{Verification Algorithm}

Verification MUST proceed in the following order:

\begin{enumerate}[leftmargin=*]
    \item Validate signature using issuer public key
    \item Validate expiration timestamp
    \item Validate issuance timestamp
    \item Enforce nonce replay protection
    \item Apply local trust policy
\end{enumerate}

Verification MUST fail if any step fails.

\section{Trust Store Requirements}

\begin{itemize}[leftmargin=*]
    \item Trust stores MUST be locally configurable
    \item Trust stores MAY be empty
    \item No mandatory global root of trust SHALL exist
\end{itemize}

\section{Agent-to-Agent Handshake}

Handshake procedure:

\begin{enumerate}[leftmargin=*]
    \item Exchange Identity Certificates
    \item Perform verification algorithm
    \item Validate replay nonce
    \item Negotiate permitted capabilities
    \item Establish secure session context
\end{enumerate}

Handshake MUST terminate upon verification failure.

\section{Threat Model}

The protocol assumes adversaries MAY:

\begin{itemize}[leftmargin=*]
    \item Attempt to forge identity certificates
    \item Replay previously valid handshake messages
    \item Attempt capability escalation
    \item Compromise non-hardware-protected private keys
    \item Operate malicious but cryptographically valid agents
\end{itemize}

The protocol assumes adversaries CANNOT:

\begin{itemize}[leftmargin=*]
    \item Break Ed25519 cryptographic primitives
    \item Forge signatures without private key compromise
    \item Override verifier-controlled trust policy
\end{itemize}

A2AX-Core guarantees cryptographic authenticity, not behavioral integrity.

\section{Adversary Assumptions}

\textbf{Class A --- Network Adversary}

May intercept, replay, or inject traffic. Mitigated via signature validation, expiration enforcement, and nonce replay protection.

\textbf{Class B --- Malicious Certified Agent}

Possesses a valid certificate but behaves dishonestly. Mitigated through verifier-controlled trust policy.

\textbf{Class C --- Compromised Key Holder}

Private key material has been exposed. Mitigated via expiration limits and optional short-lived certificates. Global revocation is out of scope.

\textbf{Class D --- Centralization Actor}

Attempts to impose mandatory trust anchors or ecosystem control. Mitigated through structural neutrality and locally configurable trust stores.

\section{Security Considerations}

Implementations MUST protect private keys.

Replay protection MUST be enforced.

Clock synchronization SHOULD be considered.

Hardware-backed key storage is RECOMMENDED where available.

\section{Privacy Considerations}

Capability declarations SHOULD follow data minimization principles.

Identifiers SHOULD avoid embedding personally identifiable information.

\section{Extensibility}

Extensions MUST NOT alter core verification semantics.

Future capability namespaces MAY be registered via documented extension processes.

\section{Conformance}

An implementation conforms to this specification if it:

\begin{itemize}[leftmargin=*]
    \item Implements the required Identity Certificate structure
    \item Implements the full Verification Algorithm
    \item Enforces Trust Store requirements
    \item Correctly executes the A2A handshake procedure
\end{itemize}

\section{Governance and Versioning}

Protocol changes require public proposal, documented rationale, and semantic version updates.

Backward compatibility SHOULD be preserved where possible.

\section{Fork Test Guarantee}

Removal of any specific organization, service, or ecosystem reference MUST NOT break protocol functionality.

\end{document}
